\documentclass[11pt]{scrartcl}
\usepackage[T1]{fontenc}
\usepackage[a4paper, left=3cm, right=2cm, top=2cm, bottom=2cm]{geometry}
\usepackage[activate]{pdfcprot}
\usepackage[ngerman]{babel}
\usepackage[parfill]{parskip}
\usepackage[utf8]{inputenc}
\usepackage[math]{kurier}
\usepackage{amsmath}
\usepackage{amssymb}
\usepackage{xcolor}
\usepackage{epstopdf}
\usepackage{txfonts}
\usepackage{fancyhdr}
\usepackage{graphicx}
\usepackage{prettyref}
\usepackage{hyperref}
\usepackage{eurosym}
\usepackage{setspace}
\usepackage{units}
\usepackage{eso-pic,graphicx}
\usepackage{icomma}
\usepackage{pdfpages}
\usepackage{svg}
\usepackage{pgfplots}

\definecolor{darkblue}{rgb}{0,0,.5}
\hypersetup{pdftex=true, colorlinks=true, breaklinks=false, linkcolor=black, menucolor=black, pagecolor=black, urlcolor=darkblue}



\setlength{\columnsep}{2cm}


\newcommand{\arcsinh}{\mathrm{arcsinh}}
\newcommand{\asinh}{\mathrm{arcsinh}}
\newcommand{\ergebnis}{\textcolor{red}{\mathrm{Ergebnis}}}
\newcommand{\fehlt}{\textcolor{red}{Hier fehlen noch Inhalte.}}
\newcommand{\betanotice}{\textcolor{red}{Diese Aufgaben sind noch nicht in der Übung kontrolliert worden. Es sind lediglich meine Überlegungen und Lösungsansätze zu den Aufgaben. Es können Fehler enthalten sein!!! Das Dokument wird fortwährend aktualisiert und erst wenn das \textcolor{black}{beta} aus dem Dateinamen verschwindet ist es endgültig.}}
\newcommand{\half}{\frac{1}{2}}
\renewcommand{\d}{\, \mathrm d}
\newcommand{\punkte}{\textcolor{white}{xxxxx}}
\newcommand{\p}{\, \partial}
\newcommand{\dd}[1]{\item[#1] \hfill \\}

\renewcommand{\familydefault}{\sfdefault}
\renewcommand\thesection{}
\renewcommand\thesubsection{}
\renewcommand\thesubsubsection{}


\newcommand{\themodul}{Halbleiter und Nanostrukturen - Charakteristik einer Vakuumanlage}
\newcommand{\thetutor}{Prof. Förster}
\newcommand{\theuebung}{Praktikum}

\pagestyle{fancy}
\fancyhead[L]{\footnotesize{C. Hansen}}
\chead{\thepage}
\rhead{}
\lfoot{}
\cfoot{}
\rfoot{}

\title{\themodul{}, \theuebung{}, \thetutor}


\author{Christoph Hansen, Christian große Börding \\ {\small \href{mailto:chris@university-material.de}{chris@university-material.de}} }

\date{}


\begin{document}

\maketitle

Dieser Text ist unter dieser \href{http://creativecommons.org/licenses/by-nc-sa/4.0/}{Creative Commons} Lizenz veröffentlicht.

\textcolor{red}{Ich erhebe keinen Anspruch auf Vollständigkeit oder Richtigkeit. Falls ihr Fehler findet oder etwas fehlt, dann meldet euch bitte über den Emailkontakt.}

\tableofcontents


\newpage


\section{Versuchsaufbau}

In diesem Versuch geht es darum eine Vakuumanlage zu vermessen. Dabei sollen charakteristische Daten wie das effektive Saugvermögen der Pumpen und die Leckrate bestimmt werden. \\
Zu diesem Zweck hatten wir einen Pumpenstand aus einer Drehschieberpume, die mit einer Turbomolekularpumpe gekoppelt war. Zur Druckmessung wurde ein 2 in 1 Druckmessgerät verwendet, damit alle erreichbaren Druckbereiche abgedeckt sind. \\
Zu Anfang wurde die ganze Anordnung gesäubert und dann zusammengebaut. Schematisch sah der Aufbau so aus:


Bild!!!


\section{Versuchsdurchführung}

Zuerst erzeugen wir mit der Drehschieberpumpe ein Vorvakuum im Bereich von ca $\unit[2 \cdot 10^{-2}]{mbar}$, anschließend schalten wir die Turbomolekularpumpe hinzu. Nach erreichen des Enddrucks sperren wir die Kammer mehrfach ab bestimmen darüber die Leckrate. Danach schalten wir die Turbomolekurlarpumpe ab warten bis der Enddruck der Drehschieberpumpe erreicht wird. Daraus können wir Saugvermögen bestimmen.








\end{document}